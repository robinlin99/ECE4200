
\documentclass[11pt]{article}

\usepackage{fullpage}
\usepackage{amsmath,amssymb,amsthm,amsfonts,latexsym,bbm,xspace,graphicx,float,mathtools,
verbatim, xcolor} 
\PassOptionsToPackage{hyphens}{url}\usepackage{hyperref}
\newcommand{\new}[1]{\textcolor{red}{#1}}
%\usepackage{psfig}

\newcommand{\future}[1]{\textcolor{red}{#1}}

\newcommand{\hP}{\hat P}
\newcommand{\hp}{\hat p}

\newcommand{\Dk}{\Delta_k}
\newcommand{\Px}{P(x)}
\newcommand{\Qx}{Q(x)}
\newcommand{\Nx}{N_x}

\newcommand{\Py}{P(y)}
\newcommand{\Qy}{Q(y)}
\newcommand{\Pml}{P_{ML}}
\newcommand{\Pmlx}{\Pml(x)}
\newcommand{\Pbeta}{P_{\beta}}
\newcommand{\Pbetax}{\Pbeta(x)}


\newcommand{\dTV}[2]{d_{TV} (#1,#2)}
\newcommand{\dKL}[2]{D(#1||#2)}
\newcommand{\chisq}[2]{\chi^2(#1,#2)}
\newcommand{\eps}{\varepsilon}

\newcommand{\nPepsp}[1]{n^*(#1, \eps)}
\newcommand{\nPeps}{\nPepsp{\cP}}


\newcommand{\sumX}{\sum_{x\in\cX}}

\newcommand{\Bpr}[1]{Bern(#1)}

\newenvironment{problem}[2][Problem]{\begin{trivlist}
\item[\hskip \labelsep {\bfseries #1}\hskip \labelsep {\bfseries #2.}]}{\end{trivlist}}

% Theorem-like environments

\newtheorem{Theorem}{Theorem}
\newtheorem{Theorem*}{Theorem}

\newtheorem{Claim}[Theorem]{Claim}
\newtheorem{Claim*}[Theorem]{Claim}
\newtheorem{Corollary}[Theorem]{Corollary}
\newtheorem{Conjecture}[Theorem]{Conjecture}
\newtheorem{CounterExample*}{$\overline{\hbox{\bf Example}}$}
\newtheorem{Definition}[Theorem]{Definition}
\newtheorem{Example}[Theorem]{Example}
\newtheorem{Example*}[Theorem]{Example}
\newtheorem{Exercise}[Theorem]{Exercise}
\newtheorem{Intuition*}[Theorem]{Intuition}
\newtheorem{Joke*}[Theorem]{Joke}
\newtheorem{Lemma}[Theorem]{Lemma}
\newtheorem{Lemma*}[Theorem]{Lemma}
\newtheorem{Open problem}[Theorem]{Open problem}
\newtheorem{Proposition}[Theorem]{Proposition}
\newtheorem{Property}[Theorem]{Property}
\newtheorem{Question}[Theorem]{Question}
\newtheorem{Question*}[Theorem]{Question}
\newtheorem{Remark}[Theorem]{Remark}
\newtheorem{Result}[Theorem]{Result}
\newtheorem{Fact}[Theorem]{Fact}
\newtheorem{Condition}[Theorem]{Condition}

\newcommand{\ed}{\stackrel{\mathrm{def}}{=}}
%\newcommand{\edef}{\stackrel{\mathrm{def}}{=}}
\def \Paren#1{{\left({#1}\right)}}

\newcommand{\probof}[1]{\Pr\Paren{#1}}


\newtheorem{theorem}{Theorem}
\newtheorem{proposition}[theorem]{Proposition}
\newtheorem{corollary}[theorem]{Corollary}
%\newtheorem*{corollary*}{Corollary}
\newtheorem{assumption}[theorem]{Assumption}
\newtheorem{lemma}[theorem]{Lemma}
%\newtheorem*{lemma*}{Lemma}
\newtheorem{conjecture}[theorem]{Conjecture}
\newtheorem{example}[theorem]{Example}
\newtheorem{definition}[theorem]{Definition}
\newtheorem{claim}[theorem]{Claim}

\newcommand{\Xon}{X_1^n}


\newcommand{\prob}{{\rm Pr}}
\newcommand{\Probof}[1]{\prob\left(#1\right)}



\newcommand{\ignore}[1]{}

% Equation formatting

\newcommand{\spreqn}[1]{{\qquad\text{#1}\qquad}}

% Blackboard fonts
\newcommand{\II}{\mathbb{I}} % Added by Theertha on April 16th 2013.
\newcommand{\EE}{\mathbb{E}}
\newcommand{\CC}{\mathbb{C}}
\newcommand{\NN}{\mathbb{N}}
\newcommand{\QQ}{\mathbb{Q}}
\newcommand{\RR}{\mathbb{R}}
\newcommand{\ZZ}{\mathbb{Z}}
\newcommand{\PP}{\mathbb{P}}

% Number sets

\newcommand{\complex}{\CC}
\newcommand{\integers}{\ZZ}
\newcommand{\naturals}{\NN}
\newcommand{\positives}{\PP}
\newcommand{\rationals}{\QQ}
\newcommand{\reals}{\RR}

\newcommand{\realsge}{{\reals_{\ge}}}
\newcommand{\realsp}{\reals^+}
\newcommand{\integersp}{\integers^+}
\newcommand{\integerss}[1]{\integers_{\ge{#1}}}

% boldface

\def \ba     {{\bf a}}
\def \bx     {{\bf x}}
\def \by     {{\bf y}}

\def \bA     {{\bf A}}
\def \bB     {{\bf B}}
\def \bC     {{\bf C}}
\def \bD     {{\bf D}}
\def \bF     {{\bf F}}
\def \bG     {{\bf G}}
\def \bL     {{\bf L}}
\def \bQ     {{\bf Q}}
\def \bR     {{\bf R}}
\def \bS     {{\bf S}}
\def \bT     {{\bf T}}
\def \bX     {{\bf X}}
\def \bY     {{\bf Y}}
\def \bZ     {{\bf Z}}

% caligraphics

\def \cA     {{\cal A}}
\def \cB     {{\cal B}}
\def \cC     {{\cal C}}
\def \cD     {{\cal D}}
\def \cE     {{\cal E}}
\def \cF     {{\cal F}}
\def \cG     {{\cal G}}
\def \cH     {{\cal H}}
\def \cI     {{\cal I}}
\def \cK     {{\cal K}}
\def \cL     {{\cal L}}
\def \cM     {{\cal M}}
\def \cN     {{\cal N}}
\def \cO     {{\cal O}}
\def \cP     {{\cal P}}
\def \cQ     {{\cal Q}}
\def \cR     {{\cal R}}
\def \cS     {{\cal S}}
\def \cT     {{\cal T}}
\def \cU     {{\cal U}}
\def \cV     {{\cal V}}
\def \cW     {{\cal W}}
\def \cX     {{\cal X}}
\def \cY     {{\cal Y}}
\def \cZ     {{\cal Z}}

% vectors

\def \vct#1{{\overline{#1}}}

\def \vcta  {{\vct a}}
\def \vctb  {{\vct b}}
\def \vctq  {{\vct q}}
\def \vcts  {{\vct s}}
\def \vctu  {{\vct u}}
\def \vctv  {{\vct v}}
\def \vctx  {{\vct x}}
\def \vcty  {{\vct y}}
\def \vctz  {{\vct z}}
\def \vctp  {{\vct p}}

\def \vctV  {{\vct V}}
\def \vctX  {{\vct X}}
\def \vctY  {{\vct Y}}
\def \vctZ  {{\vct Z}}

\def \vctbeta  {{\vct\beta}}



% qed's --  Also consider \qedhere

\def \eqed    {\eqno{\qed}}
\def \rqed    {\hbox{}~\hfill~$\qed$}

% sequences

\def \upto  {{,}\ldots{,}}

\def \zn    {0\upto n}
\def \znmo  {0\upto n-1}
\def \znpo  {0\upto n+1}
\def \ztnmo {0\upto 2^n-1}
\def \ok    {1\upto k}
\def \on    {1\upto n}
\def \onmo  {1\upto n-1}
\def \onpo  {1\upto n+1}

% sets

\def \sets#1{{\{#1\}}}
\def \Sets#1{{\left\{#1\right\}}}

\def \set#1#2{{\sets{{#1}\upto{#2}}}}

\def \setpmo   {\sets{\pm 1}}
\def \setmpo   {\{-1{,}1\}}
\def \setzo    {\{0{,}1\}}
\def \setzn    {\{\zn\}}
\def \setznmo  {\{\znmo\}}
\def \setztnmo {\{\ztnmo\}}
\def \setok    {\{\ok\}}
\def \seton    {\{\on\}}
\def \setonmo  {\{\onmo\}}
\def \setzon   {\setzo^n}
\def \setzos   {\setzo^*}


% Set operations



\def\ignore#1{}

 

\title{Assignment Four\\ ECE 4200}
\date{}
\begin{document}
\maketitle 

\begin{itemize}
\item
Provide credit to \textbf{any sources} other than the course staff that helped you solve the problems. This includes \textbf{all students} you talked to regarding the problems. 	
\item
You can look up definitions/basics online (e.g., wikipedia, stack-exchange, etc).
\item
{\bf The due date is 3/01/2020, 23.59.59 ET}. 
\item
Submission rules are the same as previous assignments.
\item
\textbf{Please write your net-id on top of every page. It helps with grading.}
\end{itemize}



\begin{problem}{1 (10 points) Different class conditional probabilities}
Consider a classification problem with features in $\RR^d$, and labels in $\{-1, +1\}$. Consider the class of linear classifiers of the form $(\overrightarrow w, 0)$, namely all the classifiers (hyper planes) that pass through the origin (or $t=0$). Instead of logistic regression, suppose the class probabilities are given by the following function, where $\overrightarrow X\in\RR^d$ are the features:
\begin{align}
P\Paren{y=+1|\overrightarrow X, \overrightarrow w} = \frac12\Paren{1+\frac{\overrightarrow w\cdot \overrightarrow X}{\sqrt{1+(\overrightarrow w\cdot \overrightarrow X)^2}}}, 
\end{align}
where $\overrightarrow w\cdot \overrightarrow X$ is the dot product between $\overrightarrow w$ and  $\overrightarrow X$. 

Suppose we obtain $n$ examples $(\overrightarrow X_i, y_i)$ for $i=1,\ldots, n$. 
\begin{enumerate}
\item 
Show that the log-likelihood function is
\begin{align}
J(\overrightarrow w) = -n\log 2 + \sum_{i=1}^{n} \log \Paren{1+ \frac{y_i (\overrightarrow w\cdot \overrightarrow X_i)}{\sqrt{1+(\overrightarrow w\cdot \overrightarrow X_i)^2}}}.
\end{align}
\item
Compute the gradient and write one step of gradient ascent. Namely fill in the blank:
\begin{align}
\overrightarrow w_{j+1} = \overrightarrow w_j + \eta\cdot \underline{\hspace{3cm}}\nonumber
\end{align}
\end{enumerate}
\end{problem}


\noindent In \textbf{Problem 2}, and \textbf{Problem 3}, we will study linear regression. We will assume in both the problems that $w^0=0$. (This can be done by translating the features and labels to have mean zero, but we will not worry about it).  For $\overrightarrow w = (w^1, \ldots, w^d)$, and $\overrightarrow X = (\overrightarrow X^1, \ldots, \overrightarrow X^d)$, the regression we want is:
\begin{align}
y = w^1 \overrightarrow X^1+\ldots + w^d \overrightarrow X^d = \overrightarrow w\cdot \overrightarrow X.
\end{align}
We considered the following regularized least squares objective, which is called as \textbf{Ridge Regression}. For $n$ examples $(\overrightarrow X_i, y_i)$,  
\[
J(\overrightarrow w, \lambda) = \sum_{i=1}^n \Paren{y_i -\overrightarrow w\cdot \overrightarrow X_i}^2 +\lambda \cdot \|\overrightarrow w\|_2^2.
\]	
 
\begin{problem}{2 (10 points) Gradient Descent for regression}\quad

\begin{enumerate}
\item 
Instead of using the closed form expression we derived in class, suppose we want to perform gradient descent to find the optimal solution for $J(\overrightarrow w)$. Please compute the gradient of $J$, and write one step of the gradient descent with step size $\eta$. 
\item
Suppose we get a new point $\overrightarrow X_{n+1}$, what will the predicted $y_{n+1}$ be when $\lambda\to\infty$? 
\end{enumerate}
\end{problem}


\begin{problem}{3 (15 points) Regularization increases training error}
In the class we said that when we regularize, we expect to get weight vectors with smaller, but never proved it. We also displayed a plot showing that the training error increases as we regularize more (larger $\lambda$). In this assignment, we will formalize the intuitions rigorously.

Let $0<\lambda_1<\lambda_2$ be two regularizer values. Let $\overrightarrow w_1$, and $\overrightarrow w_2$ be the minimizers of $J(\overrightarrow w, \lambda_1)$, and $J(\overrightarrow w, \lambda_2)$ respectively. 

\begin{enumerate}
	\item Show that $\|\overrightarrow w_1\|_2^2\ge \|\overrightarrow w_2\|_2^2$. Therefore more regularization implies smaller norm of solution!
	
	\textbf{Hint:} Observe that $J(\overrightarrow w_1, \lambda_1)\le J(\overrightarrow w_2, \lambda_1)$, and $J(\overrightarrow w_2, \lambda_2)\le J(\overrightarrow w_1, \lambda_2)$ (why?). 
	\item Show that the training error for $\overrightarrow w_1$ is less than that of $\overrightarrow w_2$. In other words, show that 
\[
\sum_{i=1}^n \Paren{y_i -\overrightarrow w_1\cdot \overrightarrow X_i}^2 \le \sum_{i=1}^n \Paren{y_i -\overrightarrow w_2\cdot \overrightarrow X_i}^2.
\]

\textbf{Hint:} Use the first part of the problem.
\end{enumerate}

\end{problem}




\begin{problem}{4 (25 points) Linear and Quadratic Regression}
Please refer to the Jupyter Notebook in the assignment, and complete the coding part in it!
You can use sklearn regression package: \url{http://scikit-learn.org/stable/modules/generated/sklearn.linear_model.Ridge.html}
\end{problem}





\end{document}
